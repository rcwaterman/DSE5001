% Options for packages loaded elsewhere
\PassOptionsToPackage{unicode}{hyperref}
\PassOptionsToPackage{hyphens}{url}
%
\documentclass[
]{article}
\usepackage{amsmath,amssymb}
\usepackage{iftex}
\ifPDFTeX
  \usepackage[T1]{fontenc}
  \usepackage[utf8]{inputenc}
  \usepackage{textcomp} % provide euro and other symbols
\else % if luatex or xetex
  \usepackage{unicode-math} % this also loads fontspec
  \defaultfontfeatures{Scale=MatchLowercase}
  \defaultfontfeatures[\rmfamily]{Ligatures=TeX,Scale=1}
\fi
\usepackage{lmodern}
\ifPDFTeX\else
  % xetex/luatex font selection
\fi
% Use upquote if available, for straight quotes in verbatim environments
\IfFileExists{upquote.sty}{\usepackage{upquote}}{}
\IfFileExists{microtype.sty}{% use microtype if available
  \usepackage[]{microtype}
  \UseMicrotypeSet[protrusion]{basicmath} % disable protrusion for tt fonts
}{}
\makeatletter
\@ifundefined{KOMAClassName}{% if non-KOMA class
  \IfFileExists{parskip.sty}{%
    \usepackage{parskip}
  }{% else
    \setlength{\parindent}{0pt}
    \setlength{\parskip}{6pt plus 2pt minus 1pt}}
}{% if KOMA class
  \KOMAoptions{parskip=half}}
\makeatother
\usepackage{xcolor}
\usepackage[margin=1in]{geometry}
\usepackage{color}
\usepackage{fancyvrb}
\newcommand{\VerbBar}{|}
\newcommand{\VERB}{\Verb[commandchars=\\\{\}]}
\DefineVerbatimEnvironment{Highlighting}{Verbatim}{commandchars=\\\{\}}
% Add ',fontsize=\small' for more characters per line
\usepackage{framed}
\definecolor{shadecolor}{RGB}{248,248,248}
\newenvironment{Shaded}{\begin{snugshade}}{\end{snugshade}}
\newcommand{\AlertTok}[1]{\textcolor[rgb]{0.94,0.16,0.16}{#1}}
\newcommand{\AnnotationTok}[1]{\textcolor[rgb]{0.56,0.35,0.01}{\textbf{\textit{#1}}}}
\newcommand{\AttributeTok}[1]{\textcolor[rgb]{0.13,0.29,0.53}{#1}}
\newcommand{\BaseNTok}[1]{\textcolor[rgb]{0.00,0.00,0.81}{#1}}
\newcommand{\BuiltInTok}[1]{#1}
\newcommand{\CharTok}[1]{\textcolor[rgb]{0.31,0.60,0.02}{#1}}
\newcommand{\CommentTok}[1]{\textcolor[rgb]{0.56,0.35,0.01}{\textit{#1}}}
\newcommand{\CommentVarTok}[1]{\textcolor[rgb]{0.56,0.35,0.01}{\textbf{\textit{#1}}}}
\newcommand{\ConstantTok}[1]{\textcolor[rgb]{0.56,0.35,0.01}{#1}}
\newcommand{\ControlFlowTok}[1]{\textcolor[rgb]{0.13,0.29,0.53}{\textbf{#1}}}
\newcommand{\DataTypeTok}[1]{\textcolor[rgb]{0.13,0.29,0.53}{#1}}
\newcommand{\DecValTok}[1]{\textcolor[rgb]{0.00,0.00,0.81}{#1}}
\newcommand{\DocumentationTok}[1]{\textcolor[rgb]{0.56,0.35,0.01}{\textbf{\textit{#1}}}}
\newcommand{\ErrorTok}[1]{\textcolor[rgb]{0.64,0.00,0.00}{\textbf{#1}}}
\newcommand{\ExtensionTok}[1]{#1}
\newcommand{\FloatTok}[1]{\textcolor[rgb]{0.00,0.00,0.81}{#1}}
\newcommand{\FunctionTok}[1]{\textcolor[rgb]{0.13,0.29,0.53}{\textbf{#1}}}
\newcommand{\ImportTok}[1]{#1}
\newcommand{\InformationTok}[1]{\textcolor[rgb]{0.56,0.35,0.01}{\textbf{\textit{#1}}}}
\newcommand{\KeywordTok}[1]{\textcolor[rgb]{0.13,0.29,0.53}{\textbf{#1}}}
\newcommand{\NormalTok}[1]{#1}
\newcommand{\OperatorTok}[1]{\textcolor[rgb]{0.81,0.36,0.00}{\textbf{#1}}}
\newcommand{\OtherTok}[1]{\textcolor[rgb]{0.56,0.35,0.01}{#1}}
\newcommand{\PreprocessorTok}[1]{\textcolor[rgb]{0.56,0.35,0.01}{\textit{#1}}}
\newcommand{\RegionMarkerTok}[1]{#1}
\newcommand{\SpecialCharTok}[1]{\textcolor[rgb]{0.81,0.36,0.00}{\textbf{#1}}}
\newcommand{\SpecialStringTok}[1]{\textcolor[rgb]{0.31,0.60,0.02}{#1}}
\newcommand{\StringTok}[1]{\textcolor[rgb]{0.31,0.60,0.02}{#1}}
\newcommand{\VariableTok}[1]{\textcolor[rgb]{0.00,0.00,0.00}{#1}}
\newcommand{\VerbatimStringTok}[1]{\textcolor[rgb]{0.31,0.60,0.02}{#1}}
\newcommand{\WarningTok}[1]{\textcolor[rgb]{0.56,0.35,0.01}{\textbf{\textit{#1}}}}
\usepackage{graphicx}
\makeatletter
\def\maxwidth{\ifdim\Gin@nat@width>\linewidth\linewidth\else\Gin@nat@width\fi}
\def\maxheight{\ifdim\Gin@nat@height>\textheight\textheight\else\Gin@nat@height\fi}
\makeatother
% Scale images if necessary, so that they will not overflow the page
% margins by default, and it is still possible to overwrite the defaults
% using explicit options in \includegraphics[width, height, ...]{}
\setkeys{Gin}{width=\maxwidth,height=\maxheight,keepaspectratio}
% Set default figure placement to htbp
\makeatletter
\def\fps@figure{htbp}
\makeatother
\setlength{\emergencystretch}{3em} % prevent overfull lines
\providecommand{\tightlist}{%
  \setlength{\itemsep}{0pt}\setlength{\parskip}{0pt}}
\setcounter{secnumdepth}{-\maxdimen} % remove section numbering
\ifLuaTeX
  \usepackage{selnolig}  % disable illegal ligatures
\fi
\usepackage{bookmark}
\IfFileExists{xurl.sty}{\usepackage{xurl}}{} % add URL line breaks if available
\urlstyle{same}
\hypersetup{
  pdftitle={Final Exam Prospectus},
  pdfauthor={HD Sheets},
  hidelinks,
  pdfcreator={LaTeX via pandoc}}

\title{Final Exam Prospectus}
\author{HD Sheets}
\date{2024-10-07}

\begin{document}
\maketitle

\subsection{Final Exam Prospectus}\label{final-exam-prospectus}

Prep for the exam by making sure you can do all these problems, and that
you understand the content of each data set.

You may want to do some research on your own to understand these data
sets a bit, and to be sure you understand the ideas in each problem.

Prepare an RMD with answers to all of these problems. You may use the
RMD and other sources of information during the exam. You can look
things up during the exam, but you cannot asking questions of another
living person, or of LLM models like ChatGPT or Claude.

Turn in your prepared RMD with all answers prepared and annotated at the
start of the exam. The preparation is 40\% of the exam grade.

Then you will get 3 randomly chosen questions from this set of questions
to answer on the exam during the one-hour exam window. The exam will be
60\% of the grade.

Note-If you have a problem wrong on the preparation section, that will
cause you to lose more points on the in-class exam as well, just be
aware of that and be sure you understand the preparation material.

The questions during the live exam will be modified, so while you can
(and should) cut and paste answers from your preparation RMD, you will
need to modify your answers to reflect the changes in the exam question

I might alter the exam question by

a.) changing the data set b.) changing what variables I want to see
plotted, or in a table c.) changing ranges of values

I will use the Quiz function in Canvas to randomly give you three
problems, cut and paste these into an RMD and then use your prepared
code to answer them. You can just paste the altered problems right into
your prepared RMD if that helps.

I will answer questions about the problem statements, but will not tell
you if you have a problem correct or not.

Most of the questions are derived from homework and/or PairProgrammming
examples from Modules 5 to 7, but earlier material may be included

Module 8 content is not covered on this exam.

This exam thus has a ``take home preparation'' portion and a ``live''
portion. Due to the sheer volume of material in the class and the rapid
pace, this format will reward extensive preparation work, and lower the
impact of work during the one hour exam window. At the same time, you do
have to be able to execute the code quickly, demonstrating that you
understand how it works.

\section{Exam Time}\label{exam-time}

The exam will open at 5 pm EST October 17 and close at 9pm October 18,
you must complete the exam in a one-hour long block within that time
frame

Note: I have be available on Zoom from 7 to 9 pm EST on Thursday Oct 17
and 7-9 pm EST on Friday Oct 18, so that you can jump on zoom if you
have to ask me questions. I strongly urge you to take the exam during
one of the times when I am online. I cannot guarantee you will be able
to ask me questions at other times during the exam window. If you feel
you will want to be able to ask questions, try to plan on taking the
exam between 7-9 pm EST on Thursday or Friday.

\section{Problem 1 EXAM}\label{problem-1-exam}

\begin{verbatim}
                       Health Status
\end{verbatim}

Health Excellent Very Good Good Fair Poor Total Coverage No 456 727 854
385 99 2,521 Yes 4,201 6,246 4,820 1,634 578 17,479 Total 4,657 6,973
5,674 2,019 677 20,000

a.) What is the probability that a single individual drawn at random has
poor health and does not have Health Insurance? What would be the
standard error on this estimate of a proportion?

\emph{This cohort represents 99 of the 20000 subjects.}

\begin{Shaded}
\begin{Highlighting}[]
\NormalTok{n}\OtherTok{=}\DecValTok{20000}
\NormalTok{p}\OtherTok{=}\DecValTok{99}\SpecialCharTok{/}\NormalTok{n}
\NormalTok{p}
\end{Highlighting}
\end{Shaded}

\begin{verbatim}
## [1] 0.00495
\end{verbatim}

\emph{The probability is about 0.495\%}

\begin{Shaded}
\begin{Highlighting}[]
\NormalTok{SE}\OtherTok{=}\NormalTok{((p}\SpecialCharTok{*}\NormalTok{(}\DecValTok{1}\SpecialCharTok{{-}}\NormalTok{p))}\SpecialCharTok{/}\NormalTok{n)}\SpecialCharTok{\^{}}\FloatTok{0.5}
\NormalTok{SE}
\end{Highlighting}
\end{Shaded}

\begin{verbatim}
## [1] 0.0004962609
\end{verbatim}

\emph{The standard error is about 0.0005}

b.) What is the probability of having Poor health overall?

\begin{Shaded}
\begin{Highlighting}[]
\DecValTok{677}\SpecialCharTok{/}\DecValTok{20000}
\end{Highlighting}
\end{Shaded}

\begin{verbatim}
## [1] 0.03385
\end{verbatim}

\emph{About 3.4\%}

c.) what is the probability of not having health coverage?

\begin{Shaded}
\begin{Highlighting}[]
\DecValTok{2521}\SpecialCharTok{/}\DecValTok{20000}
\end{Highlighting}
\end{Shaded}

\begin{verbatim}
## [1] 0.12605
\end{verbatim}

\emph{About 12.6\%}

d.) What is the probability of having poor health given that a person
has health coverage?

\emph{P(PO \textbar{} HC) = P(PO and HC)/P(HC)}

\begin{Shaded}
\begin{Highlighting}[]
\NormalTok{P\_hc}\OtherTok{=}\DecValTok{17479}\SpecialCharTok{/}\DecValTok{20000}
\NormalTok{P\_po\_and\_hc}\OtherTok{=}\DecValTok{578}\SpecialCharTok{/}\DecValTok{20000}
\NormalTok{P\_po\_given\_hc }\OtherTok{=}\NormalTok{ P\_po\_and\_hc}\SpecialCharTok{/}\NormalTok{P\_hc}
\NormalTok{P\_po\_given\_hc}
\end{Highlighting}
\end{Shaded}

\begin{verbatim}
## [1] 0.03306825
\end{verbatim}

\emph{About 3.3\%}

e.) Are having poor health and having health coverage independent?

\emph{To determine this, let's look at the proportion of people without
health coverage at each health level. If these variables were mutually
exclusive, I would expect there to be a dramatically smaller proportion
of people with excellent health and no health coverage.}

\begin{Shaded}
\begin{Highlighting}[]
\NormalTok{excellent}\OtherTok{=}\DecValTok{456}\SpecialCharTok{/}\DecValTok{4657}
\NormalTok{very\_good}\OtherTok{=}\DecValTok{727}\SpecialCharTok{/}\DecValTok{6973}
\NormalTok{good}\OtherTok{=}\DecValTok{854}\SpecialCharTok{/}\DecValTok{5674}
\NormalTok{fair}\OtherTok{=}\DecValTok{385}\SpecialCharTok{/}\DecValTok{2019}
\NormalTok{poor}\OtherTok{=}\DecValTok{99}\SpecialCharTok{/}\DecValTok{677}

\FunctionTok{cat}\NormalTok{(}\StringTok{"Excellent Proportion: "}\NormalTok{, }\FunctionTok{round}\NormalTok{(excellent}\SpecialCharTok{*}\DecValTok{100}\NormalTok{,}\DecValTok{2}\NormalTok{),}\StringTok{"\%}\SpecialCharTok{\textbackslash{}n}\StringTok{"}\NormalTok{, }
    \StringTok{"Very Good Proportion: "}\NormalTok{, }\FunctionTok{round}\NormalTok{(very\_good}\SpecialCharTok{*}\DecValTok{100}\NormalTok{,}\DecValTok{2}\NormalTok{),}\StringTok{"\%}\SpecialCharTok{\textbackslash{}n}\StringTok{"}\NormalTok{,}
    \StringTok{"Good Proportion: "}\NormalTok{, }\FunctionTok{round}\NormalTok{(good}\SpecialCharTok{*}\DecValTok{100}\NormalTok{,}\DecValTok{2}\NormalTok{),}\StringTok{"\%}\SpecialCharTok{\textbackslash{}n}\StringTok{"}\NormalTok{,}
    \StringTok{"Fair Proportion: "}\NormalTok{, }\FunctionTok{round}\NormalTok{(fair}\SpecialCharTok{*}\DecValTok{100}\NormalTok{,}\DecValTok{2}\NormalTok{),}\StringTok{"\%}\SpecialCharTok{\textbackslash{}n}\StringTok{"}\NormalTok{,}
    \StringTok{"Poor Proportion: "}\NormalTok{, }\FunctionTok{round}\NormalTok{(poor}\SpecialCharTok{*}\DecValTok{100}\NormalTok{,}\DecValTok{2}\NormalTok{),}\StringTok{"\%}\SpecialCharTok{\textbackslash{}n}\StringTok{"}\NormalTok{,}
    \AttributeTok{sep=}\StringTok{""}
\NormalTok{    )}
\end{Highlighting}
\end{Shaded}

\begin{verbatim}
## Excellent Proportion: 9.79%
## Very Good Proportion: 10.43%
## Good Proportion: 15.05%
## Fair Proportion: 19.07%
## Poor Proportion: 14.62%
\end{verbatim}

\emph{It appears that having poor health and health care could be
independent, as there is a relatively uniform, maybe even slightly
normal distribution, of health proportions by coverage status. The poor
proportion represents about the same proportion as the good health
category, and is only about 5\% different from the remaining
proportions.}

\section{Problem 2 EXAM}\label{problem-2-exam}

Written answers.

Suppose we have the following situations

2.1.) I wish my friend a happy birthday, but it is not her birthday.

2.2.) I forget to wish my friend a happy birthday on her birthday.

a.) Which of 2.1 and 2.2 is a false positive?

\emph{2.1 is a false positive.}

b.) Which of 2.1 and 2.2 is a false negative?

\emph{2.2 is a false negative}

c.) Which of the two errors is more of concern? What is the ``cost'' of
each mistake.

\emph{This depends on whether the critical result is positive or
negative. Let's assume that a positive is an instance of a failure. In
this case, false positives can arise from an abundance of caution, which
is often preferable in most real world scenarios. False negatives,
however, would be very dangerous in this scenario, as they would be a
failure that was not identified, and could reach the customer, which
could tarnish reputation. In the opposite case, if a pass is a positive,
the false positive would be the more dangerous error.}

If we are running a test of a new marketing campaign, of adds A and B.
We decide we want a significance level of 2.5\% for our sample size of
60 people in each group.

d.) Are we likely to make a Type 1 error?

\emph{A type 1 error occurs when the null hypothesis is rejected when it
is true. A 2.5\% significance level means there is a 2.5\% chance that
the null hypothesis has been rejected when it is true, or there is a
2.5\% chance of a type 1 error, making it unlikely.}

e.) Are we likely to make a Type 2 error?

\emph{A type 2 error is more likely than type 1, as the type 2 error
happens when the null hypothesis is accepted, even if there is a
measurable effect. Increasing the significance level reduces the risk of
type 1 error, but proportionally increases the risk of type 2 error.}

f.) What two options are there to reducing the Type 2 error? What are
the drawbacks of each of these approaches.

Source:
\url{https://corporatefinanceinstitute.com/resources/data-science/type-ii-error/\#}:\textasciitilde:text=1.,the\%20power\%20of\%20a\%20test.

\emph{Option 1: Reduce the significance level, which could lead to more
type 1 errors.} \emph{Option 2: Increase the sample size. This increases
the ability to detect differences in the hypothesis test. This could
lead to false conclusions, as small effects are more significant.}

\section{Problem 6 EXAM}\label{problem-6-exam}

Load the data set ``iris'' from the r datasets, it is a standard data
set.

\begin{Shaded}
\begin{Highlighting}[]
\FunctionTok{data}\NormalTok{(iris)}
\FunctionTok{head}\NormalTok{(iris)}
\end{Highlighting}
\end{Shaded}

\begin{verbatim}
##   Sepal.Length Sepal.Width Petal.Length Petal.Width Species
## 1          5.1         3.5          1.4         0.2  setosa
## 2          4.9         3.0          1.4         0.2  setosa
## 3          4.7         3.2          1.3         0.2  setosa
## 4          4.6         3.1          1.5         0.2  setosa
## 5          5.0         3.6          1.4         0.2  setosa
## 6          5.4         3.9          1.7         0.4  setosa
\end{verbatim}

a.) What test would you use to test the hypothesis that the Sepal.Width
was more variable in setosa than in versicola?

\emph{An F-test would be used.}

b.) Run this test in R. Did the hypothesis hold?

\begin{Shaded}
\begin{Highlighting}[]
\FunctionTok{var.test}\NormalTok{(iris[iris}\SpecialCharTok{$}\NormalTok{Species}\SpecialCharTok{==}\StringTok{"setosa"}\NormalTok{, }\DecValTok{2}\NormalTok{], iris[iris}\SpecialCharTok{$}\NormalTok{Species}\SpecialCharTok{==}\StringTok{"versicolor"}\NormalTok{, }\DecValTok{2}\NormalTok{])}
\end{Highlighting}
\end{Shaded}

\begin{verbatim}
## 
##  F test to compare two variances
## 
## data:  iris[iris$Species == "setosa", 2] and iris[iris$Species == "versicolor", 2]
## F = 1.4592, num df = 49, denom df = 49, p-value = 0.1895
## alternative hypothesis: true ratio of variances is not equal to 1
## 95 percent confidence interval:
##  0.828080 2.571444
## sample estimates:
## ratio of variances 
##           1.459233
\end{verbatim}

\emph{The p-value is greater than 0.05, therefore we cannot reject the
null hypothesis, implying the hypothesis does not hold.}

c.) Does the species variable predict differences in Petal.Width? What
test would you use to test this hypothesis?

\emph{An ANOVA test would be used to test this hypothesis.}

\begin{Shaded}
\begin{Highlighting}[]
\FunctionTok{summary}\NormalTok{(}\FunctionTok{aov}\NormalTok{(Petal.Width }\SpecialCharTok{\textasciitilde{}}\NormalTok{ Species, iris))}
\end{Highlighting}
\end{Shaded}

\begin{verbatim}
##              Df Sum Sq Mean Sq F value Pr(>F)    
## Species       2  80.41   40.21     960 <2e-16 ***
## Residuals   147   6.16    0.04                   
## ---
## Signif. codes:  0 '***' 0.001 '**' 0.01 '*' 0.05 '.' 0.1 ' ' 1
\end{verbatim}

\emph{The species variable strongly predicts differences in Petal.Width.
This is indicated by the near 0 p-value, rejecting the null hypothesis.}

\end{document}
